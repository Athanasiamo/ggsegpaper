% !TeX root = RJwrapper.tex
\title{Visualisation of Brain Statistics with R-package ggseg}
\author{by Athanasia M. Mowinckel, Didac Vidal Piñeiro}

\maketitle

\abstract{%
(150 words) An abstract of less than 150 words.
}

% Any extra LaTeX you need in the preamble

\hypertarget{introduction}{%
\subsection{Introduction}\label{introduction}}

Neuroscientific analysis usually requires the use of multiple softwares
to analyse, visualise, and summarise data. This often makes the process
of preparing results for publication laborious, as results must be
exported and imported in various formats. In neuroimaging, images of
probands brain as collected and merged together to provide three
dimensional representations. Much of the neuroimaging analyses done are
done not on the individal voxel (i.e.~3-dimentional pixels of brain
images) level, but rather on pre-defined brain segmentations, called
brain atlases. These brain atlases are plentiful and are different ways
of segmenting the brain into functionally or structurally similar
regions, and the use of these is wide-spread, as these atlases provide
larger meaningful divisions of the brain. While neuroimaging analyses on
the voxel-level are usually computed by special software for such
analyses, analyses of brain atlas data is usually done in standard
statistical software, like R \citep{R}.

A key part of understanding and disseminating analysis results, is the
visualiasion of these in a meaningful way. With regards to results from
brain atlas analyses, it is most meaningfully represented if projected
onto a representation of the brain, rather than other standard types of
charts. Each atlas has its own labelling depending on what is meaningful
for the type of segmentation it is based on, and as such for the reader
to fully understand a bar chart with atlas labels, they need to be very
familiar with the location of each label to create a clear comprehension
of the results. A projection directly onto a brain shape, eases the
readability of the results for the reader, and provides clear references
even if the used atlas is unfamiliar.

While there are several tools that aid R-users in plotting neuroimaging
data directly through R using the grammar of graphics as implemented in
ggplot2(REF), such as ggBrain(REF) and ggneuro(REF)(see
\href{https://neuroconductor.org/}{neuroconductor} for compiled
neuroimaging packages for R), these are based on plotting imaging files,
not results from analyses of brain atlases. We here introduce the
ggseg-package for visualising results from brain atlas analyses. The
ggseg-package was developed to create templates that others might use to
project their brain atlas results on to. It is based on the grammar of
graphics of ggplot2 using polygons, and while its plotting functions are
the what users are drawn to, it is the pre-compiled number of datasets
for different brain atlases that provides the real functionality needed
for visualisation of brain atlas results.

\hypertarget{rationale}{%
\subsection{Rationale}\label{rationale}}

\hypertarget{brain-atlas-selection}{%
\subsection{Brain atlas selection}\label{brain-atlas-selection}}

dkt, jhu, glasser, yeo7, yeo17, aseg, mid sagittal.

\hypertarget{ggseg3d---the-plotly-surface-plot}{%
\subsection{ggseg3d - the plotly surface
plot}\label{ggseg3d---the-plotly-surface-plot}}

\hypertarget{just-keep-this-in-for-now-so-we-remember-to-check}{%
\subsection{Just keep this in for now, so we remember to
check}\label{just-keep-this-in-for-now-so-we-remember-to-check}}

This file is only a basic article template. For full details of
\emph{The R Journal} style and information on how to prepare your
article for submission, see the
\href{https://journal.r-project.org/share/author-guide.pdf}{Instructions
for Authors}.

\bibliography{RJreferences}


\address{%
Athanasia M. Mowinckel\\
Center for Lifespan Changes in Brain and Cognition, Univeristy of Oslo\\
Postboks 1094 Blindern\\ 0317 Oslo\\ Norway\\
}
\href{mailto:a.m.mowinckel@psykologi.uio.bo}{\nolinkurl{a.m.mowinckel@psykologi.uio.bo}}

\address{%
Didac Vidal Piñeiro\\
Center for Lifespan Changes in Brain and Cognition, Univeristy of Oslo\\
Postboks 1094 Blindern\\ 0317 Oslo\\ Norway\\
}
\href{mailto:d.v.pineiro@psykologi.uio.no}{\nolinkurl{d.v.pineiro@psykologi.uio.no}}

